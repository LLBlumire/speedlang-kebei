\chapter{Phonology}

\section{Consonant}

\begin{table}[H]
    \centering
    \begin{tabu}{r | c c c c c}
                    &   Labial                  &   Alveolar 
                    &   Palatal                 &   Velar
                    &   Glottal                 \\\hline
                    
        Nasal       &   \broad{m} \orth{m}      &   \broad{n} \orth{n}
                    &   \broad{\xJ} \orth{\nU}  &   \broad{\xN} \orth{q}
                    &                           \\
                    
        Plosive     &   \broad{p} \orth{b}      &   \broad{t} \orth{d}
                    &   \broad{c} \orth{\dU}    &   \broad{k} \orth{g}
                    &   \broad{\gs} \orth{'}    \\
                    
        Ejective    &   \broad{\pE} \orth{p}    &   \broad{\tE} \orth{t}
                    &   \broad{\cE} \orth{\tU}  &   \broad{\kE} \orth{k}
                    &                           \\
        
        Fricative   &   \broad{f} \orth{f}      &   \broad{s} \orth{s} 
                    &   \broad{\xC} \orth{\sU}  &   \broad{x} \orth{x}
                    &   \broad{h} \orth{h}      \\
        
    \end{tabu}
    \caption{Consonant Chart}
    \label{tab:consc}
\end{table}


\section{Vowel}

\begin{table}[H]
    \centering
    \begin{vowel}[plain]
    \putcvowel{\broad{i} \orth{i}}{1}
    \putcvowel{\broad{u} \orth{u}}{8}
    \putcvowel{\broad{\xE} \orth{e}}{3}
    \putcvowel{\broad{\xO} \orth{o}}{6}
    \end{vowel}
    \caption{Vowel Chart}
    \label{tab:vowelc}
\end{table}

All four vowels also distinguish phonemically rhotacisation, and there are 4 diphthongs: \broad{\xE{}i}, \broad{\xE{}u}, \broad{\xO{}i}, \broad{\xO{}u}. 

Diphthongs are written by placing the two vowels adjacent to each other, rhotacisation is written by having the second vowel be an \orth{e}.

This gives a total of 12 distinct vowels.

Vowels also can have two tones, high or low, where high is marked with diaeresis, in diphthongs a single overdot over each of the vowels, i.e. \orth{\eD}, \orth{\uO\eO}, \orth{\oO\uO}

\subsection{Phonotactics}

CVF where C is consonant, V is vowel, and F is fricative consonant.