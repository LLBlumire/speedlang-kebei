\chapter{Challenges}


\section{Language Showcase}

The entirety of this document serves as an answer to challenge 1.

\section{Syntax Tests}

For the purposes of these syntax tests, it will be assumed that the speaker, and any people references, are a caste 3 citizen.

\subsection{Bring your friends with you.}

\broad{\kE{}i\HI{}h\xN{}\xE{}hx\xO{}us \tE{}i\HI{} mi\xN{}\xE{}i mo\xC\pE{}\xE{}uto n\xE{}is\xO{}n\xE{}\HI{}xc\xO{} \tE{}i \tE{}u \kE{}\uR{}f\xO{}u \xN{}\xE{} \tE{}i}

This gloss demonstrates the use of the progress aspect to construct temporal clauses. The verb phrase preceding it is said to be the ``case'' of what follows in it's aspect. Here, preceding is `you go to here' and proceeding is `exist with your friends'

\enumsentence{
    \shortex{7}
    { k\iD{}hqeh=xous
    & t\iD{}
    & miqei 
    & mo\sU{}peudo 
    & neison\eD{}x=\dU{}o
    & ti 
    & tu 
    }
    { 2\gl{v}\bs{}go=\gl{supc}>\gl{nonh} 
    & \gl{a}\bs{}\gl{sg}\bs{}2.\gl{elc}
    & \gl{p}\bs{}\gl{sg}\bs{}here
    & \gl{prog}
    & 1\gl{v}\bs{}exist=\gl{supc}>\gl{supc}
    & \gl{s}\bs{}\gl{sg}\bs{}2.\gl{elc} 
    & and
    }
    {}\\
    \shortex{3}
    { kuefou
    & qe 
    & ti
    }
    { \gl{s}\bs{}\gl{pl}\bs{}friend
    & of
    & \gl{sg}\bs{}2.\gl{elc}
    }
    {`Come here with your friends'}
}

\subsection*{The robot jumped onto the table.}

\broad{n\xO{}\HI{}i\HI{}hu\tE{}\xE{} \xJ{}\xE{}k\xE{}i \pE{}\xE{}u \pE{}\xE{}s h\xE{}\HI{}i\HI{}\cE{}\xO{}u}

\enumsentence{
    \shortex{5}
    { n\oO{}\iO{}hu=te
    & \nU{}egei
    & peu
    & pes
    & h\eO{}\iO{}\tU{}ou
    }
    { 2\gl{v}\bs{}jump=\gl{nonh}>\gl{nonh}
    & \gl{p}\bs{}\gl{sg}\bs{}table
    & location
    & top
    & \gl{a}\bs{}\gl{sg}\bs{}robot
    }
    {'Robot jumps onto table'}
}

This shows the usage of \orth{peu} to mark relative locations to objects, here, the tabletop.

\subsection*{All of her clothes --- tops, leggings, and undergarments --- were packed away in a storage unit.}

\broad{c\xE{}i mu\HI{}t\xE{}\HI{}u\HI{}x\xO{}us c\xE{}p\xO{}ih x\xE{}uspu\HI{}\Rhotic{} \pE{}u m\xE{}ih \xN{}\xE{} \cE{}\xE{}uni nu p\eR{}\xC{}\cE{}\xE{}\HI{}\Rhotic \tE{}u \xC{}i\xE{}hp\xO{}\HI{}i\HI{} \tE{}u \xN{}\eR{}n\xE{}\HI{}u\HI{}}

\enumsentence{
    \shortex{9}
    { \dU{}ei
    & m\uD{}d\eO{}\uO{}=xous
    & \dU{}epoih
    & xeusb\uO{}\eO{}
    & pu
    & meih
    & qe
    & \tU{}euni
    & nu
    }
    { \gl{ret}
    & 3\gl{v}\bs{}store=\gl{supc}>\gl{nonh}
    & \gl{r}\bs{}\gl{sg}\bs{}storage\_unit
    & \gl{t}\bs{}\gl{pl}\bs{}clothes
    & count
    & all
    & of
    & \gl{sg}\bs{}3.\gl{elc}
    & info
    }
    {}\\
    \shortex{5}
    { bee\sU{}\tU{}\eO{}\eO{}
    & tu
    & \sU{}iehb\oO{}\iO{}
    & tu
    & qeen\eO{}\uO{}
    }
    { \gl{t}\bs{}\gl{pl}\bs{}shirt
    & and
    & \gl{t}\bs{}\gl{pl}\bs{}trousers
    & and
    & \gl{t}\bs{}\gl{pl}\bs{}undergarment
    }
    {`All of their clothes --- tops, leggings, and trousers --- had been stored in a storage unit'}
}

Demonstrates the use of an aspect without a preceding temporal verb phrase, as well as the use of the info connective used to provide additional context.

\subsection*{Come when you are called.}

\broad{mu\HI{}n\xE{} \tE{}i c\xE{}i \kE{}i\HI{}h\xN{}\xE{}h mi\xN{}\xE{}i}

\enumsentence{
    \shortex{5}
    { m\uD{}ne
    & ti
    & \dU{}ei 
    & k\iD{}hqeh
    & miqei
    }
    { 2\gl{v}\bs{}call\_for
    & \gl{p}\bs{}\gl{sg}\bs{}2.ELC
    & \gl{ret}
    & 2\gl{v}\bs{}go
    & \gl{p}\bs{}\gl{sg}\bs{}here
    }
    {`When you are called come here'}
}

\subsection*{The stars shone.}

\broad{si\pE{}\xE{}\HI{}u\HI{} kuh\xO{}i}

\enumsentence{
    \shortex{2}
    { sip\eO{}\uO{}
    & guhoi
    }
    { 1\gl{v}\bs{}shine
    & \gl{s}\bs{}\gl{sg}\bs{}stars
    } 
    {`Stars Shine'}
}

\section{Kinship System}

The kinship system is described in table \ref{tab:kinship}

\section{Time}

This challenge was not completed

\section{Writing System}

The writing system is designed for quick digital expression through the computers communication system, it is a featural system based on a 3 by 2 grid and binary phonetic notation. 

You can think of the writing system as the following cells

\begin{center}
\includegraphics[width=0.3\textwidth]{images/orthog}
\end{center}

Red represents the place of articulation, green represents the manner of articulation, and blue is a special point used to mark vowels. The four places of articulation are numbered front to back

\begin{itemize}[label={}]
    \item $000$ --- Labial
    \item $001$ --- Alveolar
    \item $010$ --- Palatal
    \item $011$ --- Velar
    \item $100$ --- Glottal
\end{itemize}

These aranged clockwise from the top left.

The four manners of articulation are numbered

\begin{itemize}[label={}]
    \item $00$ --- Nasal
    \item $01$ --- Plosive
    \item $10$ --- Ejective
    \item $11$ --- Fricative
\end{itemize}

If the final cell is low, it is a consonant, if the final cell is high, then the pattern of cells determines the vowel according to the following pattern (read left to right, top to bottom). The very first cell represents a high or low tone.

\begin{itemize}[label={}]
    \item $010101$ --- i
    \item $010111$ --- u
    \item $001101$ --- e
    \item $011011$ --- o
\end{itemize}

Meaning the text `Stars shine' is written in script as

\includegraphics[width=0.6\textwidth]{images/sunshine}