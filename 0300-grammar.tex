\chapter{Grammar}

There are 2 grammatical categories in \langname{}. These are Nouns and Verbs; Adwords (Adjectives, Adverbs) behave as verbs.

\section{Nouns}

\subsection{Plural / Dual}

Plurality and Duality are marked by rhotacisation of vowels. If a word is singular it takes no rhotacisation on it's vowels, if it is dual then the final vowel on the word stem is rhotacised. If it is plural then the initial vowel on the word stem is rhotacised. One syllable words show only singular/plural distinction, if a vowel is a diphthong no distinction is made.

\begin{itemize}[label={}]
    \item \orth{\sU{}e} \broad{\xC\xE} ``Adult Man''
    \item \orth{\sU{}ee} \broad{\xC\eR} ``Adult Men''
    \item \orth{dux'usu} \broad{dux\gs{}usu} ``One Flatbread''
    \item \orth{dux'usue} \broad{dux\gs{}us\uR{}} ``Two Flatbreads''
    \item \orth{duex'usu} \broad{d\uR{}x\gs{}usu} ``Many Flatbreads''
\end{itemize}

There are a number of irregularities to this pattern howerver, for example the word \orth{guhoi} means `Stars' plural and not \orth{guehoi} and it's singular form is \orth{peufif} `Start; Point'.

\subsection{Semantic Roles}

Verbs use tones to make for valency, the valency of the sentence determines the tone patterns available to mark for varying semantic roles.

The general morphosyntactic alignment of \langname{} is ergative-absolutive-secundative. This is graphically shown in Figure \ref{fig:ergabssec} where Red is a low tone, green is a high initial tone, and blue is a high final tone.

\begin{figure}[H]
    \centering
    \includegraphics[width=0.5\textwidth]{images/erg-abs-sec}
    \caption{Ergative-Absolutive-Secundative Alignment}
    \label{fig:ergabssec}
\end{figure}

For monosylabic words in trivalent sentences, \orth{'ux} is appended if the word is an agent, and \orth{su} is appended if the word is a patient.

Topicalisation is done by backing the topic of a sentence.


\subsection{Pronouns}

Pronouns are determined by caste. They fall into 8 main groups. A special pronoun that is used to refer to caste 0 individuals, a special pronoun that is used to refer to caste 7 individuals and non human entities, and 6 pronouns that are determined by relative caste, separated by lesser, equal, greater, and by the legality of any children you could have.

\begin{table}[H]
\centering
\caption{Pronoun Table}
\label{tab:pronoun}
\begin{tabu}{r|cccccccc}
  & \multirow{2}{*}{Caste 0} & \multicolumn{2}{c}{Greater} & \multicolumn{2}{c}{Equal} & \multicolumn{2}{c}{Lesser}    & \multirow{2}{*}{Caste 7 / Non Human} \\
  &                          & Legal          & Illegal    & Legal         & Illegal   & Legal               & Illegal &                                      \\\hline
1 & kofku                    & xi             & xe         & poudo         & 'e        & \nU{}oif\sU{}ef & hehmei  & beis                                 \\
2 & houqeu                   & be\dU{}eif   & sou        & ti            & 'uhou     & 'e'euf              & 'ehfeu  & gi                                   \\
3 & boipi                    & 'i             & \tU{}eu  & \tU{}euni   & \dU{}e  & qogu                & keneu   & neu                                 
\end{tabu}
\end{table}

\subsection{Noun Phrase Construction}

Noun phrases are right branching appending information to nouns. There are a number of words that can be used as a part of this. Largely involves some form of dependency where one noun depends upon another noun with a relationship. The most trivial of these is possession ``i.e. my flatbread'' can be analysed as ``flatbread of mine''. This is done with the word \orth{qe}. i.e.

Another word used in noun phrase construction is \orth{tu} ``and''. Marking both the noun phrase to it's left and the noun phrase to it's right as filling an equal role in the sentence. Both should be marked tonally as the same semantic role.

To assign number to a noun, the word \orth{pu} ``count'' is used followed by a number. Numbers are base 8 and numbers above multiples of $10_8$ are written by placing the multiple before the word, this continues for higher exponents. i.e. $3 700 106_8$ is \orth{xei-fiqeu-bi-xefmei-poife-gi}.


\begin{table}[H]
    \centering
    \begin{tabular}{r | l}
        0 & gei \\
        1 & \nU{}of \\
        2 & dogo \\
        3 & xei \\
        4 & pouse \\
        5 & tihno \\
        6 & gi \\
        7 & bi \\
        10 & mei \\
        100 & poife \\
        1000 & difeix \\
        10000 & \tU{}eido\sU{} \\
        100000 & xefmei \\
        1000000 & fiqeu
    \end{tabular}    
    \label{tab:number}
    \caption{Number System}
\end{table}

i.e. `6 men' is  \orth{\sU{}e pu gi\sU{}o}. You can also use the word \orth{meih} to mean `all' with \orth{pu}

Different parts of an object are specified with \orth{peu} ``location''. For example ``Tabletop'' \orth{\nU{}egei peu pes}.

If additional information is being provided about a noun, such as defining it or giving a description, then the word \orth{nu} ``info'' is used.


\section{Verbs}

\subsection{Valency}

Verbs use tone to mark valency. The following rules are used for the number of the valency

\begin{enumerate}
    \item Final syllable high tone, append \orth{go} if not enough syllables
    \item Initial syllable high tone, append \orth{he} if not enough syllables
    \item Intial and Final syllable high tone, append \orth{me} if not enough syllables
\end{enumerate}

\subsection{Polypersonal Agreement}

Verbs mark polypersonal agreement. This is done through agglutination of the following parts.

\begin{table}[H]
    \centering
    \begin{tabular}{c|c c c}
                    &   Superior    &   Inferior    &   Non Human   \\\hline
        Superior    &   \dU{}o      &   \tU{}oi     &   xous        \\
        Inferior    &   hoif        &   tei         &   peu         \\
        Non Human   &   toi         &   'uh         &   te          \\
    \end{tabular}
    \caption{Polypersonal Agreement, Source on Left, Target on Top}
    \label{tab:polyag}
\end{table}

\subsubsection{Derivational Morphology}

The \orth{go} prefix denotes the undoing of an action. i.e. \orth{ken\eO{}\uO{}\dU{}o} ``I go'' vs \orth{goken\eO{}\uO{}\dU{}o} ``I return''.


\section{Temporal Constructions}

Temporal constructions are placed before the verb in a sentence. The default unmarked temporal construction is the present imperfective. There are no specific words for tense, however a time or date or relative construction (last week) can be given followed by an aspect marker (imperfective, perfective, prospective, retrospective, completive, inceptive, progressive, punctual).

\begin{itemize}[label={}]
    \item imperfective --- \nullset{}
    \item perfective --- xeu
    \item prospective --- kuxqoi
    \item retrospective --- \dU{}ei
    \item completive --- soi
    \item inceptive --- \tU{}oi
    \item progressive --- mo\sU{}peudo
    \item punctual --- doideinoi
\end{itemize}